% !TEX root = ../Projektdokumentation.tex

\clearpage
\section{Projektplanung} 
\label{sec:Projektplanung}


\subsection{Projektphasen}
\label{sec:Projektphasen}

Die Projektphase begann am 13.03.2017 und endete am 02.06.2016.
Die tägliche Arbeitszeit betrug 7 Stunden 48 Minuten mit 30 Minuten Mittagspause.
Die Projektarbeit fand nicht durchgängig statt, da Betriebsinterne Aufgaben und Ereignisse berücksichtigt werden müssen.

\tabelle{Zeitplanung}{tab:Zeitplanung}{Zeitplanung}

Eine detailliertere Zeitplanung findet sich im \Anhang{app:Zeitplanung}.

%\subsection{Abweichungen vom Projektantrag}
%\label{sec:AbweichungenProjektantrag}
%
%Die Vereinheitlichung des Fortschrittsberechnung und -ausgabe sind aufgrund der derzeitigen Architektur und  des Designs noch nicht möglich

\subsection{Ressourcenplanung}
\label{sec:Ressourcenplanung}

\begin{itemize}
\renewcommand{\labelitemi}{}
\renewcommand{\labelitemii}{\textbullet}
\renewcommand{\labelitemiii}{\normalfont\bfseries\textendash}
	\itemd{Hardware}{}
		\begin{itemize}
			\item Büroarbeitsplatz mit Tisch, Stuhl, Stromanschlüsse
			\item Arbeitsmaschine 1 mit Windows7
			\item Arbeitsmaschine 2 mit Kartenleser und Zugang zum Entwicklungsnetz der \DRV
		\end{itemize}
	\itemd{Software}{}
		\begin{itemize}
			\item Visual Studio Professional 2013 + .NET-Framework (mindestens v2.0)
			\item \ac{SDK}s für Visual Studio
			\begin{itemize}
				\itemd{CrytalReports for VisualStudio}{Framework zum Erzeugung von PDFs aus Dokumenten}
				\itemd{GhostScript}{Framework zum Drucken von Dokumenten}
				\itemd{TX-Spell-dotNet-0300}{Framework zur Rechtschreibprüfung von Dokumenten}		\itemd{TX-Text-Control-dotNet-1900}{Framework zur Darstellung von Dokumenten}
			\end{itemize}
			\item MiKTeX - Distribution des Textsatzsystems TEX
			\item TeXStudio - Entwicklungsumgebung für \LaTeX
			\item microTool inStep - Projektverwaltungstool für Arbeitsmaschine 2
		\end{itemize}
	\itemd{Personal}{}
		\begin{itemize}
			\item Projektbetreuer zur Unterstützung
		\end{itemize}
\end{itemize}

\subsection{Entwicklungsprozess}
\label{sec:Entwicklungsprozess}

Die ausgewählte Vorgehensweise ist das Wasserfall-Modell\footnote{Wasserfall-Modell nach \cite{wasserfallRoyce}}. Es ist konventionell vorgesehen, dass alle Schritte im Wasserfall-Modell sequentiell zu bearbeiten sind, \dahe Kein Schritt darf übersprungen werden und Ein Neustart oder Abbruch des Projektes bei nicht erfolgreichen Abschluss eines Schrittes. In der IT-Branche wird jedoch meist statt einem Projektneustart eher jeweilige Schritte zurückgegangen, da ein Neustart wirtschaftlich gesehen ein vieler größer Aufwand wäre.

\begin{enumerate}
	\itemd{Systemanforderungen:}{ Alle Anforderungen, die selbst nicht direkt das Software-Produkt betreffen, werden zunächst festgelegt. Dazu zählen:}
		\begin{itemize}
			\item Preis
			\item Verfügbarkeit
			\item Sicherheitsaspekte
			\item Dokumentation
		\end{itemize}
	
	\itemd{Softwareanforderungen:}{Alle Anforderung an die Software selber werden definiert. Jegliche Funktionen, Interaktionen und Besonderheiten werden konkretisiert, so dass sich aus den Systemanforderungen und Softwareanforderungen das Lastenheft ergibt.}
	
	\itemd{Analyse:}{Anforderungen aus Lastenheft und Ist-Zustand der Situation werden analysiert, so dass diese in ein Pflichtenheft umformuliert werden können. Die Wirtschaftlichkeit eines Projektes wird ebenfalls hier geprüft.}

	\itemd{Entwurf:}{Es wird das Datenmodell, die Architektur und die Schnittstellen zu anderen Anwendungen herausgearbeitet. Zwischenergebnisse können sein:}
		\begin{itemize}
			\item \ac{ERM}
			\item \acs{UML}-Diagramme (Klassendiagramm, Sequenzdiagramm, Anwendungsfalldiagramm \usw)
			\item Mockups zur \acs{GUI}
			\item Schnittstellen-Verzeichnis
		\end{itemize}
	
	\itemd{Implementierung:}{Umsetzung der Funktionalitäten nach Pflichtenheft und Entwurf in eine lauffähige Anwendung.}
	
	\itemd{Test/Qualitätssicherung:}{Es wird nach der Implementierungsphase die Software auf Fehler, Schwachstellen und Unstimmigkeiten überprüft. weitverbreitete Testmethoden:}
		\begin{itemize}
			\item Komponententests (Unit-Test)
			\item Modultests
			\item Systemtests
			\item Integrationstests
		\end{itemize}
	
	\itemd{Inbetriebnahme:}{letzter Schritt in der Softwareentwicklung. Bei fehlerloser Überprüfung kann die Anwendung abgenommen werden und in Produktion gehen.}
\end{enumerate} 


