% !TEX root = ../Projektdokumentation.tex

\newpage
\section{Projektplanung} 
\label{sec:Projektplanung}


\subsection{Projektphasen}
\label{sec:Projektphasen}

Die (Vorbereitungs-)Projektphase begann am 17.10.2016 und endete am 16.12.2016.
Die tägliche Arbeitszeit betrug 7 Stunden 48 Minuten mit 30 Minuten Mittagspause.
Die Projektarbeit fand nicht durchgängig statt, da Betriebsinterne Ereignis berücksichtigt werden müssen.

\tabelle{Zeitplanung}{tab:Zeitplanung}{Zeitplanung}

Eine detailliertere Zeitplanung findet sich im \Anhang{app:Zeitplanung}.

\subsection{Abweichungen vom Projektantrag}
\label{sec:AbweichungenProjektantrag}

Entwurfsphase mit 5 Stunden ist dazu gekommen, da eine Planung her muss.

Implementierungsphase ist daher um gleiche Zeitspanne verringert.  % AAbweichungen vervollständigen

\subsection{Ressourcenplanung}
\label{sec:Ressourcenplanung}

Benötigte Ressourcen: % AAuflisten?
\begin{itemize}
	\item Büro mit Tisch, Stuhl, Stromanschlüsse
	
	\item Arbeitsmaschine 1 mit Windows7
	\item dotNET-Framework + Visual Studio für Arbeitsmaschine 1
	\item Projektspezifische \acs{SDK}s für Visual Studio
	
	\item Arbeitsmaschine 2 mit Entwicklungsnetz-Zugang + Kartenleser
	\item microTool inStep - (Projektverwaltungstool) für Arbeitsmaschine 2
	
	\item Projektbetreuer zur Unterstützung
\end{itemize}

\subsection{Entwicklungsprozess}
\label{sec:Entwicklungsprozess}

Die ausgewählte Vorgehensweise ist das Wasserfall-Modell\footnote{Wasserfall-Modell nach W. Royce}.

\begin{enumerate}
	\itemd{Systemanforderungen:}{ Alle Anforderungen, die selbst nicht direkt das Software-Produkt betreffen, werden zunächst festgelegt. Dazu zählen:}
		\begin{itemize}
			\item Preis
			\item Verfügbarkeit
			\item Sicherheitsaspekte
			\item Dokumentation
		\end{itemize}
	
	\itemd{Softwareanforderungen:}{Alle Anforderung an die Software selber werden definiert. Jegliche Funktionen, Interaktionen und Besonderheiten werden konkretisiert, so dass sich aus den Systemanforderungen und Softwareanforderungen das Lastenheft ergibt.}
	
	\itemd{Analyse:}{Anforderungen aus Lastenheft und Ist-Zustand der Situation werden analysiert, so dass diese in ein Pflichtenheft umformuliert werden können. Die Wirtschaftlichkeit eines Projektes wird ebenfalls hier geprüft.}

	\itemd{Entwurf:}{Es wird das Datenmodell, die Architektur und die Schnittstellen zu anderen Anwendungen herausgearbeitet. Zwischenergebnisse:}
		\begin{itemize}
			\item \ac{ERM}
			\item \acs{UML}-Diagramme (Klassendiagramm, Sequenzdiagramm, Anwendungsfalldiagramm \usw)
			\item Mockups zur \acs{GUI}
			\item Schnittstellen-Verzeichnis
		\end{itemize}
	
	\itemd{Implementierung:}{Umsetzung der Funktionalitäten nach Pflichtenheft und Entwurf in eine lauffähige Anwendung.}
	
	\itemd{Test/Qualitätssicherung:}{Es wird nach der Implementierungsphase die Software auf Fehler, Schwachstellen und Unstimmigkeiten überprüft. weitverbreitete Testvorgänge:}
		\begin{itemize}
			\item Komponententests (Unit-Test)
			\item Modultests
			\item Systemtests
			\item Integrationstests
		\end{itemize}
	
	\itemd{Inbetriebnahme:}{letzter Schritt in der Softwareentwicklung. Bei fehlerloser Überprüfung kann die Anwendung abgenommen werden und in Produktion gehen.}
\end{enumerate} 

Alle Schritte im Wasserfall-Modell sind sequentiell zu bearbeiten, \dahe Kein Schritt darf vorgezogen werden oder übersprungen werden. Es dürfen nur Phasen zurückgesprungen werden, Wenn eine Phase nicht abgeschlossenen werden kann.

%ProjektPhasen
%\begin{itemize}
%	\item In welchem Zeitraum und unter welchen Rahmenbedingungen (\zB Tagesarbeitszeit) findet das Projekt statt?
%	\item Verfeinerung der Zeitplanung, die bereits im Projektantrag vorgestellt wurde.
%\end{itemize}

%ProjektAbweichungen
%\begin{itemize}
%	\item Sollte es Abweichungen zum Projektantrag geben (\zB Zeitplanung, Inhalt des Projekts, neue Anforderungen), müssen diese explizit aufgeführt und begründet werden.
%\end{itemize}

%RessourcenPlanung
%\begin{itemize}
%	\item Detaillierte Planung der benötigten Ressourcen (Hard-/Software, Räumlichkeiten \usw).
%	\item \Ggfs sind auch personelle Ressourcen einzuplanen (\zB unterstützende Mitarbeiter).
%	\item Hinweis: Häufig werden hier Ressourcen vergessen, die als selbstverständlich angesehen werden (\zB PC, Büro). 
%\end{itemize}

%Entwicklungsprozess
%\begin{itemize}
%	\item Welcher Entwicklungsprozess wird bei der Bearbeitung des Projekts verfolgt (\zB Wasserfall, agiler Prozess)?
%\end{itemize}