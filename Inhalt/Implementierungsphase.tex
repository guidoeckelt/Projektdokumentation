% !TEX root = ../Projektdokumentation.tex
\clearpage
\section{Implementierungsphase} 
\label{sec:Implementierungsphase}

\subsection{Implementierung der Geschäftslogik}
\label{sec:ImplementierungGeschaeftslogik}


Das Sequenzdiagramm zu diesem Prozess kann im \Anhang{app:Sequenzdiagramm} eingesehen werden.

\subsection{Verwendete Entwurfsmuster}
\label{sec:Entwurfsmuster}

Für die Instanziierung der einzelnen konkreten Objekte der Schnittstelle \gqq{IGlobalesAktualisierenAnlage} verwendete ich das Fabrik-Entwurfsmuster nach folgenden Schema.
\begin{figure}
%	\centering
	\includegraphicsKeepAspectRatio{Klassendiagramm-Fabrik.png}{1}
	\caption{Klassendiagramm zur Verwendung des Farbik-Entwurfsmusters}
\end{figure}


%\begin{itemize}
%	\item Beschreibung des Vorgehens bei der Umsetzung/Programmierung der entworfenen Anwendung.
%	\item \Ggfs interessante Funktionen/Algorithmen im Detail vorstellen, verwendete Entwurfsmuster zeigen.
%	\item Quelltextbeispiele zeigen.
%	\item Hinweis: Wie in Kapitel~\ref{sec:Einleitung}: \nameref{sec:Einleitung} zitiert, wird nicht ein lauffähiges Programm bewertet, sondern die Projektdurchführung. Dennoch würde ich immer Quelltextausschnitte zeigen, da sonst Zweifel an der tatsächlichen Leistung des Prüflings aufkommen können.
%\end{itemize}
%
%\paragraph{Beispiel}
%Die Klasse \texttt{Com\-par\-ed\-Na\-tu\-ral\-Mo\-dule\-In\-for\-ma\-tion} findet sich im \Anhang{app:CNMI}.  


\Zwischenstand{Implementierungsphase}{Implementierung}

%Implementierung der Geschäftslogik