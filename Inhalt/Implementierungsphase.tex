% !TEX root = ../Projektdokumentation.tex
\clearpage
\section{Implementierungsphase} 
\label{sec:Implementierungsphase}

\subsection{Implementierung der Geschäftslogik}
\label{sec:ImplementierungGeschaeftslogik}

Der Prüfungsalgorithmus zur Bestimmung der Asynchronität der Dokumente ist in der Klasse  \gqq{AnlagenPruefungsBearbeiter} implementiert.
Jede Aktualisierungslogik ist in der Methode \gqq{Erzeuge} der jeweiligen konkreten I\-Glo\-ba\-les\-Ak\-tu\-a\-li\-sie\-ren\-An\-la\-ge-Klasse implementiert.

\subsection{Verwendete Entwurfsmuster}
\label{sec:Entwurfsmuster}

Für die Instanziierung der einzelnen konkreten Objekte der Schnittstelle \gqq{IGlobalesAktualisierenAnlage} verwendete ich das Fabrik-Entwurfsmuster, welches nach dem Schema in der Abbildung \ref{fig:FactoryPattern} angewendet wird.

\begin{figure}[h]
	\centering
	\includegraphicsKeepAspectRatio{Klassendiagramm-Fabrik.png}{1}
	\caption{Klassendiagramm zur Verwendung des Farbik-Entwurfsmusters}
	\label{fig:FactoryPattern}
\end{figure}

\Zwischenstand{Implementierungsphase}{Implementierung}
