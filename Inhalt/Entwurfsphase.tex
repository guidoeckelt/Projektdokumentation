% !TEX root = ../Projektdokumentation.tex

\section{Entwurfsphase} 
\label{sec:Entwurfsphase}

\subsection{Zielplattform}
\label{sec:Zielplattform}

\ac{CBP-AD} ist als Desktopanwendung in \acs{VB}.NET, wie in \ref{sec:Projektumfeld}: \nameref{sec:Projektumfeld} erwähnt, auf Basis des .NET-Framework Version 2.0 implementiert.
Sie wird als x86-Anwendung für Windows-7 entwickelt.


\subsection{Benutzeroberfläche}
\label{sec:Benutzeroberflaeche}

Die \acs{GUI} der \ac{CBP-AD} ist, wie in \ref{sec:Projektumfeld}: \nameref{sec:Projektumfeld} erwähnt, als Windows-Forms-Oberfläche realisiert. Das Aussehen der Steuerelemente ist durch \ac{CI} vordefiniert.
Für diese Funktionalität sind keine Änderung an der \acs{GUI} nötig, da die Oberfläche zur Fortschrittsausgabe schon in bereits bestehenden Klassen implementiert wurde.


\subsection{Datenmodell}
\label{sec:Datenmodell}

Die Funktionalität \gqq{\titel} beinhaltet keine Speicherung von Entitäten, daher ist kein neues Datenmodell erforderlich.

\clearpage
\subsection{Geschäftslogik}
\label{sec:Geschaeftslogik}

Das Kommando stellt die Schnittstelle zur Oberfläche zum Aufrufen der Funktionalität dar. Dieses instantiiert dann seinen Prozess, der dann wiederum seinen Ueberarbeiter instantiiert. Der Ueberarbeiter ist das zentrale Logikgerüst dieser Funktonalität. Er delegiert Prüfungs- und Aktualisierungsaufgaben an seine internen Bearbeiter, die dann je nach Aufgabe diese selbst implementieren oder an entsprechende Klasse weitergeben. Die Ausführung und Fortschrittsausgabe dieses Prozess ist in den Klassen JobgruppeAllgemein und JobAllgemein gekapselt.

\begin{figure}[htb]
	\centering
	\includegraphicsKeepAspectRatio{Klassendiagramm-Ausschnitt.png}{1}
	\caption{Ausschnitt des Klassendiagramms}
\end{figure}

Das vollständige Klassendiagramm kann im \Anhang{app:Klassendiagramm} eingesehen werden.
Der Ablauf des Prozesses kann im \Anhang{app:Sequenzdiagramm} eingesehen werden.

\subsection{Maßnahmen zur Qualitätssicherung}
\label{sec:Qualitaetssicherung}

Die Funktionalität "Globales Aktualisieren" wird durch die Komponententests auf korrekte Ausführung mit der Entwicklungsumgebung geprüft. Für die Einführung in die nächste Release-Version der \CBPAD werden nochmal Verbundtests, bei denen alle Funktionalitäten in Verbindung mit den anderen Anwendungen der \acs{CBP}\xspace geprüft werden, durch die Abteilung durchgeführt.


\Zwischenstand{Entwurfsphase}{Entwurf}