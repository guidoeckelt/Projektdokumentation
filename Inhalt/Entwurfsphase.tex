% !TEX root = ../Projektdokumentation.tex

\section{Entwurfsphase} 
\label{sec:Entwurfsphase}

\subsection{Zielplattform}
\label{sec:Zielplattform}

\ac{CBP-AD} ist als Desktopanwendung in \acs{VB}.NET auf Basis des .NET-Framework Version 2.0 implementiert.
Sie wird als x86-Anwendung für zurzeit Windows-XP und Windows-7 entwickelt.


\subsection{Entwurf der Benutzeroberfläche}
\label{sec:Benutzeroberflaeche}

Die \acs{GUI} ist als WindowsForms-Oberfläche realisiert. Das Aussehen der Benutzerelemente ist durch \ac{CI} vordefiniert

\paragraph{Beispiel}
Die Abbildung zum Entwurf des Fortschrittsfensters findet sich im \Anhang{app:Entwuerfe}. 

\subsection{Datenmodell}
\label{sec:Datenmodell}

Die Funktionalität \gqq{\titel} beinhaltet keine Speicherung von Entitäten, daher ist kein Datenmodell erforderlich.


\subsection{Geschäftslogik}
\label{sec:Geschaeftslogik}

\paragraph{Beispiel}
Ein Klassendiagramm, welches die Klassen der Anwendung und deren Beziehungen untereinander darstellt kann im \Anhang{app:Klassendiagramm} eingesehen werden.

\Abbildung{Modulimport} zeigt den grundsätzlichen Programmablauf beim Einlesen eines Moduls als \ac{EPK}.
\begin{figure}[htb]
\centering
\includegraphicsKeepAspectRatio{modulimport.pdf}{0.9}
\caption{Prozess des Einlesens eines Moduls}
\label{fig:Modulimport}
\end{figure}



\subsection{Maßnahmen zur Qualitätssicherung}
\label{sec:Qualitaetssicherung}

Die Funktionalität "Globales Aktualisieren" wird durch die Komponententests auf korrekte Ausführung mit der Entwicklungsumgebung geprüft. Für die Einführung in die nächste Release-Version der \CBPAD werden nochmal Verbundtests, bei denen alle Funktionalitäten in Verbindung mit den anderen Anwendung der \CBP geprüft werden, durch die Abteilung durchgeführt.


\Zwischenstand{Entwurfsphase}{Entwurf}

%ZielPlatform
%\begin{itemize}
%	\item Beschreibung der Kriterien zur Auswahl der Zielplattform (\ua Programmiersprache, Datenbank, Client/Server, Hardware).
%\end{itemize}

%Architekturdesign
%\begin{itemize}
%	\item Beschreibung und Begründung der gewählten Anwendungsarchitektur (\zB \acs{MVC}).
%	\item \Ggfs Bewertung und Auswahl von verwendeten Frameworks sowie \ggfs eine kurze Einführung in die Funktionsweise des verwendeten Frameworks.
%\end{itemize}

%Entwurf der GUI
%\begin{itemize}
%	\item Entscheidung für die gewählte Benutzeroberfläche (\zB GUI, Webinterface).
%	\item Beschreibung des visuellen Entwurfs der konkreten Oberfläche (\zB Mockups, Menüführung).
% 	\item \Ggfs Erläuterung von angewendeten Richtlinien zur Usability und Verweis auf Corporate Design.
% \end{itemize}

%Datenmodell
%\begin{itemize}
%	\item Entwurf/Beschreibung der Datenstrukturen (\zB \acs{ERM} und/oder Tabellenmodell, \acs{XML}-Schemas) mit kurzer Beschreibung der wichtigsten (!) verwendeten Entitäten.
%\end{itemize}

%Geschäftslogik
%\begin{itemize}
%	\item Modellierung und Beschreibung der wichtigsten (!) Bereiche der Geschäftslogik (\zB mit Kom\-po\-nen\-ten-, Klassen-, Sequenz-, Datenflussdiagramm, Programmablaufplan, Struktogramm, \ac{EPK}).
%	\item Wie wird die erstellte Anwendung in den Arbeitsfluss des Unternehmens integriert?
%\end{itemize}

%Maßnahmen zur QA
%\begin{itemize}
%	\item Welche Maßnahmen werden ergriffen, um die Qualität des Projektergebnisses (siehe Kapitel~\ref{sec:Qualitaetsanforderungen}: \nameref{sec:Qualitaetsanforderungen}) zu sichern (\zB automatische Tests, Anwendertests)?
%	\item \Ggfs Definition von Testfällen und deren Durchführung (durch Programme/Benutzer).
%\end{itemize}

%Pflichtenheft/Datenverarbeitungskonzept
%\begin{itemize}
%	\item Auszüge aus dem Pflichtenheft/Datenverarbeitungskonzept, wenn es im Rahmen des Projekts erstellt wurde.
%\end{itemize}