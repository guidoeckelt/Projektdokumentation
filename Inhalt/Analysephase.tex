% !TEX root = ../Projektdokumentation.tex

\clearpage
\section{Analysephase} 
\label{sec:Analysephase}


\subsection{Ist-Analyse} 
\label{sec:IstAnalyse}

Die Betriebsprüfer müssen die Dokumente einzeln über den Dokumentenbaum aktualisieren. Dies kann sehr aufwendig sein, da manche Berechnungen zu Änderungen in mehreren Dokumenten führen. Das bedeutet man muss teilweise den kompletten Dokumentenbaum, siehe \Anhang{app:Screenshots}, durchgehen, um alle Dokumente aktualisieren zu können.


Diese Funktionalität ist von den Betriebsprüfern dringend erwünscht, da es eine enorme Zeitersparnis für sie ergeben würde.

\subsection{Auszug aus dem Fachkonzept}
\label{sec:AuszugFachkonzept}

\gqq{Hier wird ein Auszug aus dem Fachkonzept stehen}

\subsection{Wirtschaftlichkeitsanalyse}
\label{sec:Wirtschaftlichkeitsanalyse}

\subsubsection{\gqq{Make or Buy}-Entscheidung}
\label{sec:MakeOrBuyEntscheidung}

Da die Entwicklung der \acs{CBP-AD} ein internes Projekt der \DRV und nur eine Funktionserweiterung ist, lässt sich kein fertiges Produkt finden, dass alle Anforderungen, vor allem fachliche, erfüllt. Daher wird dieses Projekt in Eigenentwicklung von der IT-Abteilung der \DRV umgesetzt.

\subsubsection{Projektkosten}
\label{sec:Projektkosten}

Die Kosten für die Durchführung des Projekts setzen sich sowohl aus Personal-, als auch aus Ressourcenkosten zusammen.
Laut Tarifvertrag\footnote{\href{http://www.oeffentlichen-dienst.de/auszubildende/25-tarifvertrag-fuer-den-auszubildende/90-tvaoed-besonderer-teil-bbig.html}{Tarifvertrag des öffentlichen Dienstes}} verdient ein Auszubildender im dritten Lehrjahr pro Monat \eur{949,02} Brutto. 

\begin{eqnarray}
8 \mbox{ h/Tag} \cdot 220 \mbox{ Tage/Jahr} = 1.760 \mbox{ h/Jahr}\\
\eur{949,02}\mbox{/Monat} \cdot 12,9 \mbox{ Monate/Jahr} \approx \eur{12.242,36} \mbox{/Jahr}\\
\frac{\eur{12.242,36} \mbox{/Jahr}}{1.760 \mbox{ h/Jahr}} \approx \eur{6,96}\mbox{/h}
\end{eqnarray}

Es ergibt sich also ein Stundenlohn von \eur{6,96}. 
Die Durchführungszeit des Projekts beträgt 70 Stunden. Für die Nutzung von Ressourcen\footnote{Räumlichkeiten, Arbeitsplatzrechner etc.} wird 
ein pauschaler Stundensatz von \eur{15} angenommen. Für die anderen Mitarbeiter wird pauschal ein Stundenlohn von \eur{25} angenommen. 
Eine Aufstellung der Kosten befindet sich in Tabelle~\ref{tab:Kostenaufstellung}. Es ergibt sich daraus insgesamt \eur{1.697,20}.
\tabelle{Kostenaufstellung}{tab:Kostenaufstellung}{Kostenaufstellung.tex}


\subsubsection{Amortisationsdauer}
\label{sec:Amortisationsdauer}

Nach Umfragen bei den Betrieberprüfers ist eine Zeiteinsparung von 10 Minuten pro Tag wahrscheinlich. Daraus ergibt sich für jeden der 4.000 Betriebsprüfer und 220 Arbeitstagen im Jahr eine gesamte Zeiteinsparung von 
\begin{eqnarray}
4.000 \cdot 220 \mbox{ Tage/Jahr} \cdot 10 \mbox{ min/Tag} = 8.800.000 \mbox{ min/Jahr} \approx 146.667 \mbox{ h/Jahr} 
\end{eqnarray}

Dadurch ergibt sich eine jährliche Einsparung von 
\begin{eqnarray}
146.667 \mbox{h} \cdot \eur{(25 + 15)}{\mbox{/h}} = \eur{5.866.680}
\end{eqnarray}

Die Amortisationszeit beträgt also $\frac{\eur{1697,20}}{\eur{5.866.680}\mbox{/Jahr}} \approx 0,000.2 \mbox{ Jahre} \approx 2 \mbox{ Stunden}$.

\Zwischenstand{Analysephase}{Analyse}
%UseCases - entfällt wegen nur einem