% !TEX root = ../Projektdokumentation.tex

\newpage
\section{Analysephase} 
\label{sec:Analysephase}


\subsection{Ist-Analyse} 
\label{sec:IstAnalyse}

Die Betriebsprüfer müssen die Dokumente einzeln über den Dokumenttypenbaum aktualisieren. Dies kann sehr aufwendig sein, da manche Berechnungen in mehreren Dokumenten aufgelistet werden und sind diese erst ausfindig zu machen, um dann aktualisiert werden zu können.

Diese Funktionalität ist von den Betriebsprüfern dringend erwünscht, da es eine enorme Zeitersparnis für sie ergeben würde.
\subsection{Wirtschaftlichkeitsanalyse}
\label{sec:Wirtschaftlichkeitsanalyse}

\subsubsection{\gqq{Make or Buy}-Entscheidung}
\label{sec:MakeOrBuyEntscheidung}

Da die Entwicklung der \acs{CBP-AD} ein internes Projekt der \acs{DRV} ist und nur eine Funktionserweiterung ist, lässt sich kein fertiges Produkt finden, dass alle Anforderungen, vor allem fachliche, erfüllt. Daher wird dieses Projekt in Eigenentwicklung von der IT-Abteilung der \acs{DRV} umgesetzt.

\subsubsection{Projektkosten}
\label{sec:Projektkosten}

Die Kosten für die Durchführung des Projekts setzen sich sowohl aus Personal-, als auch aus Ressourcenkosten zusammen.
Laut Tarifvertrag\footnote{\href{http://www.oeffentlichen-dienst.de/auszubildende/25-tarifvertrag-fuer-den-auszubildende/90-tvaoed-besonderer-teil-bbig.html}{Link zum Tarifvertrag}} verdient ein Auszubildender im dritten Lehrjahr pro Monat \eur{949,02} Brutto. 

\begin{eqnarray}
8 \mbox{ h/Tag} \cdot 220 \mbox{ Tage/Jahr} = 1760 \mbox{ h/Jahr}\\
\eur{949.02}\mbox{/Monat} \cdot 12,9 \mbox{ Monate/Jahr} \approx \eur{12242,36} \mbox{/Jahr}\\
\frac{\eur{12242,36} \mbox{/Jahr}}{1760 \mbox{ h/Jahr}} \approx \eur{6,96}\mbox{/h}
\end{eqnarray}

Es ergibt sich also ein Stundenlohn von \eur{6,96}. 
Die Durchführungszeit des Projekts beträgt 70 Stunden. Für die Nutzung von Ressourcen\footnote{Räumlichkeiten, Arbeitsplatzrechner etc.} wird 
ein pauschaler Stundensatz von \eur{15} angenommen. Für die anderen Mitarbeiter wird pauschal ein Stundenlohn von \eur{25} angenommen. 
Eine Aufstellung der Kosten befindet sich in Tabelle~\ref{tab:Kostenaufstellung} und sie betragen insgesamt \eur{1697,20}.
\tabelle{Kostenaufstellung}{tab:Kostenaufstellung}{Kostenaufstellung.tex}


\subsubsection{Amortisationsdauer}
\label{sec:Amortisationsdauer}

Bei einer Zeiteinsparung von 10 Minuten am Tag für jeden der 4000 Betriebsprüfer und 220 Arbeitstagen im Jahr ergibt sich eine gesamte Zeiteinsparung von 
\begin{eqnarray}
4000 \cdot 220 \mbox{ Tage/Jahr} \cdot 10 \mbox{ min/Tag} = 8800000 \mbox{ min/Jahr} \approx 146667 \mbox{ h/Jahr} 
\end{eqnarray}

Dadurch ergibt sich eine jährliche Einsparung von 
\begin{eqnarray}
146667 \mbox{h} \cdot \eur{(25 + 15)}{\mbox{/h}} = \eur{5866680}
\end{eqnarray}

Die Amortisationszeit beträgt also $\frac{\eur{1697,20}}{\eur{5866680}\mbox{/Jahr}} \approx 0,0002 \mbox{ Jahre} \approx 2 \mbox{ Stunden}$.


%UseCases - entfällt wegen nur einem


\subsection{Lastenheft/Fachkonzept}
\label{sec:Lastenheft}

\paragraph{Beispiel}
Ein Beispiel für ein Lastenheft findet sich im \Anhang{app:Lastenheft}. 

%\Zwischenstand{Analysephase}{Analyse}

%Analyse
%\begin{itemize}
%	\item Wie ist die bisherige Situation (\zB bestehende Programme, Wünsche der Mitarbeiter)?
%	\item Was gilt es zu erstellen/verbessern?
%\end{itemize}

%WirtschaftlichkeitsAnalyse
%\begin{itemize}
%	\item Lohnt sich das Projekt für das Unternehmen?
%\end{itemize}

%%Make-Or-Buy-Entscheidung
%\begin{itemize}
%	\item Gibt es vielleicht schon ein fertiges Produkt, dass alle Anforderungen des Projekts abdeckt?
%	\item Wenn ja, wieso wird das Projekt trotzdem umgesetzt?
%\end{itemize}

%%ProjektKosten
%\begin{itemize}
%	\item Welche Kosten fallen bei der Umsetzung des Projekts im Detail an (\zB Entwicklung, Einführung/Schulung, Wartung)?
%\end{itemize}

%%Amortisationsdauer
%\begin{itemize}
%	\item Welche monetären Vorteile bietet das Projekt (\zB Einsparung von Lizenzkosten, Arbeitszeitersparnis, bessere Usability, Korrektheit)?
%	\item Wann hat sich das Projekt amortisiert?
%\end{itemize}

%Nutzwertanalyse
%\begin{itemize}
%	\item Darstellung des nicht-monetären Nutzens (\zB Vorher-/Nachher-Vergleich anhand eines Wirtschaftlichkeitskoeffizienten). 
%\end{itemize}

%Anwendungsfälle
%\begin{itemize}
%	\item Welche Anwendungsfälle soll das Projekt abdecken?
%	\item Einer oder mehrere interessante (!) Anwendungsfälle könnten exemplarisch durch ein Aktivitätsdiagramm oder eine \ac{EPK} detailliert beschrieben werden. 
%\end{itemize}

%Qualitätsanforderungen
%\begin{itemize}
%	\item Welche Qualitätsanforderungen werden an die Anwendung gestellt (\zB hinsichtlich Performance, Usability, Effizienz \etc (siehe \citet{ISO9126}))?
%\end{itemize}

%Lastenheft/Fachkonzept
%\begin{itemize}
%	\item Auszüge aus dem Lastenheft/Fachkonzept, wenn es im Rahmen des Projekts erstellt wurde.
%	\item Mögliche Inhalte: Funktionen des Programms (Muss/Soll/Wunsch), User Stories, Benutzerrollen
%\end{itemize}

%Zwischenstand
