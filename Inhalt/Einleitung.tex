% !TEX root = ../Projektdokumentation.tex

\section{Einleitung}
\label{sec:Einleitung}


\subsection{Projektumfeld} 
\label{sec:Projektumfeld}

Die \ac{DRV Bund} ist ein bundesweit tätiger Träger der gesetzlichen Rentenversicherung in der Bundesrepublik Deutschland
mit \ca 17.000 Mitarbeitern.\\
Zum Aufgabenfeld gehören:
\begin{itemize}
	\item Bearbeitung von Rentenanträgen und Auszahlung von Renten
	\item Überprüfung von Sozialabgaben auf Richtigkeit
	\item Beratung zu gesetzlichen Pflichten und Rechten
	\item Bearbeitung von Rehabilitationsanträgen und Inspektion der Rehabilitationseinrichtungen
\end{itemize}

Für die Überprüfung von Sozialabgaben entwickelt die IT-Abteilung der \DRV verschiedene Anwendungen für alle Träger der Deutschen Rentenversicherung, um diesen Prozess zu vereinfachen.

Die Betriebsprüfung übernimmt die Nachprüfung der Sozialabgaben für bundesweit oder international tätige, in Deutschland gemeldete Betriebe. Die Betriebsprüfer müssen dafür viele Berechnungen durchführen und verschiedenste Dokumente, Anlagen und Schreiben( nachfolgend nur noch Dokumente genannt) erstellen.

Die \ac{CBP-AD} ist eine Desktopanwendung mit der jeweilig benötigte Dokumente für Betriebsprüfungen erzeugt werden können. Diese Dokumente basieren auf Datenquellen der Desktopanwendung \ac{CBP-NB}.


\subsection{Projektziel} 
\label{sec:Projektziel}

Die Betriebsprüfer erstellen und bearbeiten Dokumente, die unter anderem mit Daten aus Berechnungen der Anwendung \acs{CBP-NB} befüllt werden. Wenn Daten verändert werden, sind diese Dokumente in einem \gqq{asynchronen} Zustand und müssen vor Weiterverwendung aktualisiert werden. Nähere Erläuterungen befinden sich im Abschnitt \ref{sec:IstAnalyse}: \nameref{sec:IstAnalyse}.

Im Hauptmenü der \acs{CBP-AD} soll ein neuer Menüeintrag bereitgestellt werden, dessen Kommando einen Aktualisierungsprozess anstößt, der alle Dokumente auf Asynchronität prüft und anschließend veraltete aktualisiert.

Für die verschiedenen Dokumenttypen gibt es zurzeit noch unterschiedliche Vorgehensweisen, wie die jeweiligen Dokumente neu erzeugt werden. Für die Funktionalität \gqq{Globales Aktualisieren} sollen nun alle Dokumenttypen auf eine einheitliche Vorgehensweise umgebaut werden.

Erzeugungsstrukturen für einige Dokumenttypen berechnen ihren Fortschritt eigenständig und geben diesen in einer eigenen Oberfläche aus. Für diese soll eine Möglichkeit der Unterdrückung dieser Fortschrittsausgabe implementiert werden, damit der Aktualisierungsprozess \gqq{Globales Aktualisieren} dies einheitlich für alle Dokumenttypen ausgeben kann.


\subsection{Projektbegründung} 
\label{sec:Projektbegruendung}

Durch diese Erweiterung wird eine Vereinheitlichung der Dokumentenaktualisierung erreicht, die zugleich eine erhebliche Vereinfachung für den Anwender mit sich bringt.


\subsection{Projektschnittstellen} 
\label{sec:Projektschnittstellen}

Die Datenquellen werden über Schnittstellen in \acs{DLL}s der \acs{CBP-NB} der \acs{CBP-AD} zur Verfügung gestellt.

Die Benutzer der Anwendung sind die Betriebsprüfer der Deutschen Rentenversicherung.


\subsection{Projektabgrenzung} 
\label{sec:Projektabgrenzung}

Dieses Projekt zur Erweiterung der \acs{CBP-AD} ist unabhängig von der Entwicklung der \ac{CBP-NB}, da nur bereits festgelegte Schnittstellen zum Datenaustausch benutzt werden und keine Änderung dieser notwendig sind.