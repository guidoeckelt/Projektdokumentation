% !TEX root = ../Projektdokumentation.tex
\section{Fazit} 
\label{sec:Fazit}

\subsection{Soll-/Ist-Vergleich}
\label{sec:SollIstVergleich}

Wie in Tabelle~\ref{tab:Vergleich} zu erkennen ist, konnte die Zeitplanung bis auf wenige Ausnahmen eingehalten werden. Die Analysephase brauchte weniger Zeit als geplant, weil die IT-Abteilung schon bereits den größten Teil der fachlichen Analyse vorbereitet hatte. Das Erstellen der Dokumentation hingegen brauchte etwas mehr Zeit als geplant, da die Verwendung von LateX zu einigen Komplikation geführt hatte.
\tabelle{Soll-/Ist-Vergleich}{tab:Vergleich}{Zeitnachher.tex}


\subsection{Lessons Learned}
\label{sec:LessonsLearned}

Durch Projekte wie dieses wird einem erst bewusst, wie wichtig Anforderungen und Spezifikationen sind, da sie zum Teil die einzigen Anhaltspunkte zur Feststellung des Entwicklungsfortschritts sind.

\subsection{Ausblick}
\label{sec:Ausblick}

Nach erfolgreichem Abschließen der Verbundtests können die Anwendungen in Produktion gehen.

Es sind keine Erweiterungen der Funktionalität geplant.
