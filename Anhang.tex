% !TEX root = Projektdokumentation.tex
\section{Anhang}

\subsection{Detaillierte Zeitplanung}
\label{app:Zeitplanung}
\tabelleAnhang{ZeitplanungKomplett}

%\input{Anhang/AnhangLastenheft.tex}
%\input{Anhang/AnhangPflichtenheft.tex}

\subsection{Klassendiagramm}
\label{app:Klassendiagramm}

Die grün umrandeten Klassen sind verwendete, bereits existierende Klassen der Code-Basis.
Alle schwarz umrandeten Klassen sind durch dieses Projekt entstanden.

\begin{figure}[htb]
\centering
\includegraphicsKeepAspectRatio{Klassendiagramm.png}{1}
\caption{Vollständiges Klassendiagramm}
\end{figure}

\subsection{Sequenzdiagramm}
\label{app:Sequenzdiagramm}
\gqq{hier könnte ihr Klassendiagramm stehen}
%\begin{figure}[htb]
%	\centering
%	\includegraphicsKeepAspectRatio{Sequenzdiagramm.png}{1}
%	\caption{Klassendiagramm}
%\end{figure}

\clearpage
\subsection{Screenshots der Anwendung}
\label{app:Screenshots}

\begin{figure}[htb]
	\centering
	\includegraphicsKeepAspectRatio{Dokumentenbaum.png}{0.7}
	\caption{Dokumentenbaum mit den verschiedenen Dokumenttypen sortiert nach Prüfgebieten}
\end{figure}

\begin{figure}[htb]
	\centering
	\includegraphicsKeepAspectRatio{Menueeintrag.png}{0.8}
	\caption{Menüeintrag zum Anstoß des Globalen Aktualisierens}
\end{figure}

\input{Anhang/AnhangDoc.tex}

\clearpage
%\subsection{Klasse: ComparedNaturalModuleInformation}
%\label{app:CNMI}
%Kommentare und simple Getter/Setter werden nicht angezeigt.
%\lstinputlisting[language=php]{Listings/cnmi.php}
%\clearpage
\subsection{Testfälle}
\label{app:Test}

\begin{figure}[htb]
	\centering
	\includegraphicsKeepAspectRatio{UnitTest.png}{1}
	\caption{Komponententest}
\end{figure}

%\lstinputlisting[language=php]{Listings/tests.php}
%\clearpage
%\begin{figure}[htb]
%\centering
%\includegraphicsKeepAspectRatio{testcase.jpg}{1}
%\caption{Aufruf des Testfalls auf der Konsole}
%\end{figure}

\subsection{Benutzerdokumentation}
\label{app:BenutzerDoku}
Ausschnitt aus der Benutzerdokumentation:

\gqq{hier könnte ihre Benutzerdokumentation stehen}
