% !TEX root = Projektdokumentation.tex

% Es werden nur die Abkürzungen aufgelistet, die mit \ac definiert und auch benutzt wurden. 
%
% \acro{VERSIS}{Versicherungsinformationssystem\acroextra{ (Bestandsführungssystem)}}
% Ergibt in der Liste: VERSIS Versicherungsinformationssystem (Bestandsführungssystem)
% Im Text aber: \ac{VERSIS} -> Versicherungsinformationssystem (VERSIS)

% Hinweis: allgemein bekannte Abkürzungen wie z.B. bzw. u.a. müssen nicht ins Abkürzungsverzeichnis aufgenommen werden
% Hinweis: allgemein bekannte IT-Begriffe wie Datenbank oder Programmiersprache müssen nicht erläutert werden,
%          aber ggfs. Fachbegriffe aus der Domäne des Prüflings (z.B. Versicherung)

% Die Option (in den eckigen Klammern) enthält das längste Label oder
% einen Platzhalter der die Breite der linken Spalte bestimmt.
\begin{acronym}[WWWWWW]
	\acro{API}{Application Programming Interface}
	\acro{HTTP}{Hypertext Transport Protocol}
	\acro{SDK}{Software Development Kit}
	\acro{ERM}{En\-ti\-ty-Re\-la\-tion\-ship-Mo\-dell}
	\acro{UML}{Unified Modeling Language}
	\acro{GUI}{Grafische Benutzeroberfläche}
	\acro{SVN}{Subversion}
	\acro{VB}{Visual Basic}
	
	\acro{DRV}{\betriebName}
	\acro{CBP}{\CBP}
	\acro{CBP-AD}{\CBPAD}
	\acro{CBP-NB}{\CBPNB}
	
	\acro{CI}{Corporate Identity}
	
	\acro{IDE}{Integrated Development Environment}
	\acro{MVC}[MVC]{Model View Controller}
	\acro{ORM}{Object-Relational Mapping}
	\acro{SQL}{Structured Query Language}
	\acro{EPK}{Ereignisgesteuerte Prozesskette}
	
	\acro{XML}{Extensible Markup Language}
	\acro{CSV}{Comma Separated Value}
	
	\acro{AO}{\textsc{Alte Oldenburger} Krankenversicherung AG}
	\acro{NatInfo}[\textsc{NatInfo}]{Natural Information System}
	\acro{Natural}[\textsc{Natural}]{Programmiersprache der Software AG}\acused{Natural}
\end{acronym}
